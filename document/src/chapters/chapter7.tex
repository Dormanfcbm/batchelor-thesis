\chapter{Conclusions and future work}
\label{chapter:conclusionfw}
This thesis presented a decentralized algorithm that focuses on forming and maintaining
a fleet of drones. The fleet has a designated leader that follows its own mission
plan and follower drones. The followers do not a have a specified mission plan 
and have the task of following the actions of the leader. Being informed
by the leader's position, speed and heading,  the followers compute their own
position in the formation and move in the desired direction. 

The \textit{Autonomous UAV Project}, of which this thesis is a part of is divided
in two components: \textit{Independent Autopilot} and \textit{Management Platform}.
The formation flight module integrates with the other modules from the \textit{
Independent Autopilot}, like the collision avoidance module, to implement an 
independent flight behavior. The proposed solution for formation flight is
based on two behaviors: \textbf{entering the formation} and \textbf{maintaining
the formation}.

The leader is coordinated by the programmed mission and is not controlled
by the formation flight algorithm. In the mission execution the leader will
engage in collision avoidance and will return to the mission execution when
the obstacle has been avoided. A follower that is controlled by the formation
flight algorithm will first try to enter in formation and when it is in the
necessary spot it will try to maintain it. If the leader moves further away, 
the follower will switch back to the \textit{entering the formation behavior}.

The main advantage of the proposed approach is the reduced quantity of information
exchange by each agent (position, speed and heading). This limited information
makes this approach a feasible real life swarm simulation, where each agent
reacts to the surrounding environment in a fully reactive manner. The current
approach is currently limited by the communication modules capacity to communicate
in real-time. If a follower is not instantly notified by the leader change in 
direction it will maintain the current heading while the leader will steer to one
direction. This can lead to a formation loss while the leader steers to one direction
and thus increasing the distance between the followers and the leader. In this
case when the follower detects that the distance has increased and it is no longer
in the accepted distance range, it will enter in the \textit{forming the formation}
mode and start to steer in the necessary direction to renter the formation.

The current approach is able to successfully maintain either a \textit{Line
Astern Formation} or a \textit{V Formation} using a fleet of 3 UAVs as long
as no communication delay is introduced. The algorithm has a time and memory
complexity of $O(1)$ which makes is a perfect candidate to run on a \textit{Raspberry
Pi Computer}. The advantage of using this \textit{Raspberry Pi} come from the
fact that it is a lightweight computer with reduced dimensions,  more specifically
$85.60$ mm x $56$ mm x $21$ mm weighting only $45$ grams. Adding new formations
can be easily done by adding new drones and geometries in the formation file
that can be seen in \labelindexref{Listing}{lst:formation-json}.

From the simulations point of view a possible improvement is represented by running
the FGFS instances in a dedicated \textit{Open Nebula} cluster  where each instance
runs on a dedicated machine what will communicate with external modules over 
a TCP connection. This architecture would provide a simulation improvement 
reducing the rendering overhead observed in the current simulations. If on 
a machine only one instance will run the communication with that machine will
be reduced greatly and the FGFS will be able to render the simulations in 
real-time.

Another future plan is to test the environment in real life conditions
using the \textit{Hirrus} drones provided by TNI. To achieve this the 
flight modules have to be modified in such a manner that it will not receive
data over a TCP socket, but instead it will use a CAN bus. Another modification
that has to be implemented is that the communication module has to format the 
messages received from the flight modules using the \textit{Hirrus} communication
format.

From an algorithmic point a view there are a couple of improvements that can 
be made. The first improvement is the integration of PID\abbrev{PID}{Proportional Integral Derivative}
controller for the heading calculations and also for speed modifications. Another
improvement would be the implementation of formations following a virtual leader.
In the current approach the formation is maintained by mimicking the behavior of 
a real drone. A virtual leader is computed as a center of mass of the formation
and the mission is defined for the overall formation. The main advantage of a
virtual leader based formation over a real leader based formation is visible 
in low altitude flights. In this situation where the fleet flies in an urban 
area,  a real leader would have to enter in collision avoidance mode often
and it will first force the follower to follow him and possibly also
engage collision avoidance. A virtual leader would not avoid any obstacle
because collisions are impossible. If a virtual leader passes through an obstacle
the followers would independently avoid the obstacle and reenter in formation. 
