\chapter{Formation Flight}
\label{chapter:formation-flight}
For flying in formation a coordinate system is needed for both positioning each 
aircraft on the world and for positioning an aircraft relative to a leader. In
aviation positioning the aircraft is usually done by GPS\abbrev{GPS}{Global
Positioning System}, but this is not the best best solution for close range formations.

\section{Coordinates Systems}
\label{sec:coordinates-systems}
The shape of \textit{Earth} has a high degree of irregularity and complexity
which makes it very hard for position. For mapping the position of
the aircraft a simpler mathematical model is needed. In geodesy this simpler
model is called \textit{figure of the Earth}, where the surface is usually
abstracted to the \textbf{Geoid} or \textbf{ellipsoids}. The Earth can be
represented as an ellipsoid with two axis where the shorter one almost matches the rotation axis \cite{gb-coordinates}.

The fact that the Earth can't be perfectly mapped on any ellipsoid, currently
there are used multiple mathematical models.
For example the GPS uses
GRS80\abbrev{GSR80}{Geodetic Reference System 1980} \cite{gsr80} and in there
United Kingdom the Airy 1830 ellipsoid is used \cite{gb-coordinates},  being 
designed to perfectly fit Britain only. The later ellipsoid is not useful in 
other parts of the world. For measuring heights, it's necessary to define an
imaginary surface that represents the \textit{zero height}. For the representation
of height the geoid model is used. The zero height shape is commonly called \textbf{sea level}.

\subsection{Latitude, Longitude,  Altitude}
\label{sub-sec:lla}
Commonly the position of the aircraft is usually done by using the longitude
  and latitude angles. These two angles pinpoint a location
on the ellipsoid that fits the globe. Thus it is mandatory to know with which
ellipsoid we are working in order to have any degree of certainty. On the globe
the North-South line are known as meridians,  and East-West lines are called 
parallels. To accurately position an aircraft in the air a third dimension is needed.
Usually the third dimension used is altitude that is measured from \textit{sea level}.
Having a 3D Cartesian System with the z-axis oriented towards the North Pole and
the x-axis and y-axis point towards the Equator, the \textbf{Zero on longitude}
would be represented by the Z-X plane and \textbf{Zero of latitude} would be 
represented by the X-Y plane, depicted in \cite{gb-coordinates} with the following
\labelindexref{Figure}{img:lla}.
% 
\fig[scale=0.5]{src/img/lla.png}{img:lla}{Latitude, longitude representation.}
\newpage

\subsection{Earth-Centered, Earth-Fixed}
\label{sub-sec:ecef}
An alternative to the angular coordinates system is the Cartesian system. Usually
satellites use a 3D Cartesian System called ECEF\abbrev{ECEF}{Earth Center,  Earth Fixed}
Coordinates. For the Earth,  the convention is to have the origin of the system
position accordingly so that it matches center of the planet. The center is defined as the barycenter 
(center of mass of the earth) and respects the following mathematical formula:

\begin{equation}
\int \overrightarrow {x}\rho dx^{3} = 0
\qquad\parbox{4.0cm}{\footnotesize$\begin{aligned} 
  \overrightarrow {x} &= \text{ position vector}
  \\[-1.0ex] \rho &= \text{ density over Earth}
  \end{aligned}$}
\label{eqn:ecef-int}
\end{equation}

If \labelindexref{Equation}{eqn:ecef-int} is not satisfied, the center should
be adjusted \cite{earth-coordinates}.

As seen in \labelindexref{Figure}{img:lla}, in ECEF the z-axis points to the North
Pole,  the x-axis points to the $0^{\circ}$ meridian and leaving the y-axis to be
set so that a right handed system is created.
\subsection{Conversion}
\label{sub-sec:lla2ecef}
Conversion between an Ellipsoid System and a Cartesian System is possible 
as long as the ellipsoid parameters are known. Modern GPS systems use 
the WGS84 ellipsoid. Coincidently the ellipsoid has the same center as the ECEF
system. The ellipsoid parameters fro WGS84 are \cite{datum}:


\begin{description}
\item [Semi-major axis] \hfill \\
$a = 6378137$
\item [Semi-minor axis] \hfill \\
$b = a(1-f) = 6356752.31424518$
\item [Ellipsoid flattening] \hfill \\
$ f = \frac{1}{298.257223563}$
\item [First Eccentricity] \hfill \\
$e = \sqrt{\frac{a^{2}-b^{2}}{a^{2}}}$
\item [Second Eccentricity] \hfill \\
$e^{'} = \sqrt{\frac{a^{2}-b^{2}}{b^{2}}}$
\end{description}

\subsubsection{LLA to ECEF}
Converting a point from LLA system into ECEF system using the metric system can
be obtained using the equations below.

\begin{equation}
\begin{cases} 
X = (N+h)\cos \varphi \cos \lambda
\\ Y = (N+h)\cos \varphi \sin \lambda
\\ Z = (\frac{b_{2}}{a_{2}}N + h) \sin \varphi
\end{cases}
\end{equation}

where

\begin {description}
\item [$\varphi$] \hfill \\ = Latitude
\item [$\lambda$] \hfill \\= Longitude
\item [$h$] \hfill \\= Height
\item [$N$] \hfill \\= Curvature radius,  = $\frac{a}{\sqrt{1-e^{2} \sin^{2} \varphi}}$
\end{description}

\subsubsection{ECEF to LLA}
Converting from ECEF to LLA is more complicated but it can be achieved  using one
of the following methods \cite{datum}:

\begin{description}
\item [Iteration for $\varphi$ and $h$] \hfill \\
This method converges quickly for  $h \ll N$ starting at $h_{0}=0$
\newline
$
\begin{cases}
\lambda = \arctan \frac{X}{Y}
\\ h_{0} = 0
\\ \varphi_{0} = \arctan \frac{Z}{p(1-e^{2})}
\\ N_{i}  = \frac{a} {\sqrt{1-e^{2}\sin^{2} \varphi_{i}}}
\\ h_{i+1} = \frac{p}{\cos \varphi_{i}} - N_{i}
\end{cases}
$
\item [Closed set formula] \hfill \\
$
\begin{cases}
\lambda = \arctan \frac{X}{Y}
\\ \varphi = \arctan \frac{Z + e^{'2}b\sin^{3} \phi}{p(-e^{2}a\cos^3 \phi)}
\\ h = \frac{p}{\cos \varphi} - N
\\ p = \sqrt{X^{2} + Y^{2}}
\\ \phi = \arctan \frac{Za}{pb}
\end{cases}
$
\end{description}

In this thesis the \textit{Closed set formula} was used to convert from ECEF
coordinates to GPS coordinates.

\section{Formation types}
\label{sec::formation-types}
Across the world human pilots are able to fly in formations that vary in both
shape and difficulty. The most used formations are usually formed by 3 
or 4 airplanes. 

In teams of 3 airplanes the most common formations are \textit{Aine Astern} and
\textit{V Formation} and can be seen in \labelindexref{Figure}{img:3form}.
% 
\fig[scale=0.5]{src/img/3form.png}{img:3form}{a) Line Astern Formation; b) V Formation}
\newpage

When having 4 airplanes the usual formations are: \textit{Line astern,  Box, 
Finger Right, Finger Left,  Echelon Right, Echelon Left} and can be seen in

\labelindexref{Figure}{img:4form1},  \labelindexref{Figure}{img:4form3} and
\labelindexref{Figure}{img:4form2}.
\fig[scale=0.25]{src/img/4form1.png}{img:4form1}{a)Line astern and b) Box}
\fig[scale=0.25]{src/img/4form2.png}{img:4form2}{c) Finger Right and d) Finger Left}
\fig[scale=0.25]{src/img/4form3.png}{img:4form3}{d) Finger Left and  e) Echelon Right}
\newpage

In this thesis we studied the {V Formation} and \textit{Line Astern}, with different
altitudes.

\section{Entering the formation}
\label{sec:formation-entering}
The process of the creation of the formation can be studied from two perspectives. If the UAVs
that need to enter the formation are situated at greater distances, the follower
must head in the leader direction with greater speed than the leader. If the 
UAVs are found inside a synchronization area (low radius sphere) they only have to
assume the correct position in formation. 

This thesis studies the behavior of a synchronization area based algorithm with
a dedicated leader. For simulating a real swarm the data shared with the other
UAVs is minimal and is composed of : \textit{latitude, longitude, altitude, speed 
and true heading}. The designated leader follows it's own mission, while
the followers have to create a flight path coherent with the leader's mission.
The leader does not inform the followers at any given time about the mission 
plan or flight path. Figure \labelindexref{Figure}{img:plane-spheres} describes
the 4 coordination spheres of an UAV. In the \textit{Emergency Space} zone, 
the Formation Flight module gives full control to the Collision Avoidance module.
In the \textit{Personal Space} zone the two modules mentioned before have equal
priorities generating a merged flight path. The previous two zones are sometimes
combined in to a single one called \textit{Collision Avoidance Zone} or 
\textit{Repel Zone}. In \textit{Synchronization zone} the follower drones have
to move in to the position specified by the formation configuration and.

\fig[scale=0.55]{src/img/plane-spheres.png}{img:plane-spheres}{Aircraft coordination spheres}
\newpage

For entering the formation, in this thesis is assumed that all the airplanes are
present in the \textit {Synchronization zone}. Entering the formation is obtained
by the following steps:

\begin {itemize}
\item The follower drone is informed by the leaders last two known position and flight
parameters.
\item Based on the formation configuration file the follower decides on which side
of the leader it should position it's self (left, right, behind).
\item The current position of the follower is computed relative to the leader.
\item If the drone is in the desired position, the drones maintains it 
by following the same heading of the leader.
\item If the drone has the move to the desired position, the drone adjusts it's
heading by incrementing or decrementing the heading with an offset computed
with the values from the configuration files.
\end {itemize}

\section{Maintaining the formation}
\label{sec:formation-maintainig}
The final step in obtaining a coherent formation for flying is maintaining
the unity of the formation once the synchronization has occurred. After
the follower has reached the desired position relative to the leader
he has to stop from steering left or right and maintain this position. This
is possible by flying with the same true heading as the leader and changing the
altitude with the same unit as the leader. Compared to a waypoint (positions on
the map that a drone has to reach), this approach is more versatile
because it doesn't have to execute turn maneuvers to reach the current waypoint.
If an drone is not able to reach o waypoint because the aeronautical model
does not allow high steering angles, the drone is forced to execute a 
circle back maneuver so that on the next pass it aligns with the waypoint. 

In this thesis we consider that all the drones have the same aeronautical model.
If the leader is able to execute a turn, the other drones are able to execute
the same maneuver. This ensures that the formation can be maintained
more easily while the leader follows the path described by the mission. 

The same flight parameters are maintained by the followers as long
as the distance between them and the leader as is maintained in the configured
range of accepted distances. If this distance is greater than the accepted ones
the followers enter in the \textit{Synchronization mode} and move to the 
needed position. If the distance is smaller then control is passed to the
\textit{Collision Avoidance} module that will ensure either that drones
execute a repel routine on the horizontal plane without changing the altitude
or that a more evasive maneuver like detach by climbing or descending is executed.

Another situation where a follower leaves the \textit{Maintain formation mode} 
and enters the \textit{Synchronization mode} is when the leader executes
a turn in the direction of the follower. In this situation the leader could 
move from the left side of the follower to the right side (or vice-versa), thus
determining the drone to steer in the same direction but with an offset added
to the heading so that it moves back to the desired location.

