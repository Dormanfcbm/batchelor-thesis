% This file contains the abstract of the thesis

\section{Versiunea in Limba Română / Romanian Version}

In ultimul deceniu s-au facut mari descoperiri in domeniul vehiculelor autonome/independente.
In acest domeniu sunt incluse si aparatele de zbor. Scopul inițial al UAV-urilor a fost unul
militar, dar recent s-a dovedit utilitatea lor si pentru aplicatii civile. Posibilitățile de 
folosire ale UAV-urilor sunt reprezentate de misiuni de 
supraveghere, cartografiere,  cautare si salvare. Cu toate acestea, 
sunt încă necesare multe imbunatățiri, precum dezvoltarea modulelor de 
zbor în formație și de evitare a obstacolelor.

% \begin{center}
Proiectul  \textit{Autonomous UAV} are ca scop dezvoltarea unei
\textit{Platforme pentru Managementul Dronelor} care este capabilă să supravegheze
și opereze o flotă de drone încă din pasul de configurare a misiunii și până la
îndeplinirea acesteia.
% \end{center}

% \begin{center}
Această lucrare se axează pe dezvoltarea unui modul responsabil cu crearea și mentinerea
unei formații strânse de zbor bazate pe o strategie de tipul \textit{urmărește
liderul}. Avand ca inspiratie modelele biologice prezente in natura, agenții sunt programați să ia
decizii automate bazate pe observarea acțiunilor celorlalți agenți.
% \end{center}


