% This file contains the abstract of the thesis

\section{Versiunea in Limba Română / Romanian Version}

% \begin{center}
In domeniul vechiculelor autonome/independete, cu precaredere aeriene, in ultimul
deceniu s-au facut mari descoperiri. Scopul initial al UAV-urilor a fost unul
militar, dar recent un interes ridicat a fost observat pentru scopuri civile.
Posibilitatile de folosire ale UAV-urilor sunt representate de misiuni de 
supraveghere, cautare si salvare si cartografiere. In ciuda ultimelor descoperiri, 
sunt inca necesare multe imbunatatiri, precum dezvoltarea modulelor de 
zbor in formatie si de evitare a obstacolelor.
% \end{center}

% \begin{center}
Proiecutl  \textit{Autonomous UAV} are ca scop dezoltarea unei
\textit{Platforme pentru Managementul Dronelor} care este capabila sa supravegheze
si opereze o flota de drone inca din pasul de configurare a misiunii si pana la
indeplinirea acesteia.
% \end{center}

% \begin{center}
Aceasta teza se axeaza pe dezvoltarea unui modul responsabil cu formarea si mentinerea
unei formatii stranse de zbor bazate pe o strategie de tipul \textit{urmareste
liderul}. Cu inspiratie din turme de animale,  agentii sunt programati sa ia
decizii automate bazate pe observarea actiunilor celorlalti agenti.
% \end{center}


