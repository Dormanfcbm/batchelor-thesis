\chapter{Related Work}
\label{chapter:related-work}

\todo{Describe related work}
A large number of articles describe the work done to solve the problem of autopilots
UAV swarms, communication and coordinate systems exists. In the following paragraphs
I will describe the main ideas and trade-offs for each described solutions.

The autopilot problem is well covered in the literature and from real life practice
it is considered that the architecture is one that is stratified. Each layer
is responsible for receiving commands from the superior layer, deciding if the
the command would put the UAV in a state of imbalance or danger in general and
forwarding the command to the layer below. If a possible imbalance state is detected
the layer either discards the received command and does not send any command to 
the next layer,  or it tweaks the command so that a incoherent state would not
be induced.

The architecture as proposed by Borges de Sousa et. al \cite{pivant} would have
the following layers:

\begin{description}
\item[Platform] The UAV vehicle with all the hardware
\item[Maneuverer controller] Controller that decides what hardware action is 
executed (roll, pitch, yaw)
\item[Vehicle supervisor] Basic autopilot capable to make simple decisions 
(setting the target speed,  setting the heading)
\item[Mission supervisor] Artificial Intelligence System that plans the mission
based on a template and the rest of the drones. It forwards the commands to the
Vehicle Supervisor for validation
\item[External controller] Human factor that can interfere between the Mission
Supervisor and Vehicle Supervisor to override or even deactivate the first one.
\end{description}

The architecture imagined by Borges de Souse can be seen in \labelindexref{Figure}{img:vechicle-arhitecture}.

\fig[scale=0.5]{src/img/vechicle-control-arhitecture.png}{img:vechicle-arhitecture}
{Vehicle Control Architecture}

\newpage
The advantages provided by this kind of architecture is the separation of
concerns for each level and the fact that the repair in case of failure can be
easily detected and fixed. The only draw-back of this approach is the fact
that in the case of poor implementation it could introduce a latency in communication
loosing the possibility of having a real time system.

In the implementation my autopilot system is similar to the one described above 
because the Hirrus drone \cite{hirrus} provided by \abbrev{TNI} already contains
a maneuver controller,vehicle supervisor  and an external controller. Thus my
system would act as a mission supervisor.

In the terms of UAV teams, Mark D. Richards and his colleagues \cite{cooperative}
identify two main groups of strategies. The first one,  called \textbf{Behavior-based
Control systems}, use a mesh of interacting high-level behaviors to perform a 
task. The second one, \textbf{Deliberative System}, acts by creating a specific
flight path for each individual UAV to follow. This second behavior is a 
generalization of the search path optimization that  Ablavsky proposes in 
  \cite{path-optimization}. Ablavsky approaches the problem by following the 
following steps:

\begin{enumerate}
\item Restrict the search area based on the mobility of the target.
\item Divide the search area in to the smallest number of sub-regions keeping in
mind the constrains of the aircraft.
\item Determine a search pattern for each sub-region with the property of assuring
full coverage with a minimized path length.
\item Combine the individual results into an optimal global path.
\end{enumerate}

The deliberative approach used by Richards is based on dividing an area for 
each UAV and generating inside the designated zone a flight path to be followed.
For achieving some degree of flexibility, Richards opted not to use an adaptive
replanning where a central controller computes a specific flight path for each
agent and and then broadcasts it to the team. The drawbacks of this approach are
represented by the fact that there would still be a single point of failure and
the fact that by the time the plan is sent to the UAV it may be already be 
deprecated. The approach that was used was a reactive one. The initial path
is computed and if an hostile condition is generated each UAV has to determine
a way to exit that state. By these means Richards managed to sweep an area
that is divided in sub-zones with different degrees of danger and even a no fly
zone. The reactive behavior was useful for avoid collision with a friendly unit
or for escaping a danger zone.

Although Richards proposes uses a long range unsynchronized flight team, in my 
implementation I used a similar approach for obtaining a synchronized formation.
