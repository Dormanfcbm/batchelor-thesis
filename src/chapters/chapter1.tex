\chapter{Introduction}
\label{chapter:intro}
\todo {write romanian version}
\section{Domain Description}
\label{sec:domain}
In the last 3 decades the aeronautic industry has focused on creating methods 
of flying that does not involve a human factor inside the airplanes, developing
solutions for unmanned flight. The necessity for advancements in UAV\abbrev{UAV}
{Unmanned Aerial Vehicle} domain is powered by the desire to keep human pilots 
out of harms way. UAV systems are useful in military missions, and high risk 
search and rescue missions. Along the military missions, UAVs can be used in 
civil context for missions like: traffic surveillance, cartography or animal 
tracking. An UAS\abbrev{UAS}{Unmanned Aerial System} or drone is a vehicle that doesn't have
a human pilot on board and can be either controlled by a RC\abbrev{RC}{Remote Control}, 
a ground control system (Control Tower) or be fully autonomous. The concepts of 
unmanned vehicles emerged a couple of years after the first mechanized flight
in 1903 by Orville and Wilbur Wright \cite{learning-to-fly}. In 1915 Nikola
Tesla had a vision about a fleet of unmanned military aircrafts and in 1919 was
developed the first UAV by Elmer Sperry,  that was used for sinking a captured
German battle ship. The first two countries that saw the high potential of unmanned vehicles were
U.S.A and Israel. In 1960 the U.S. Air Force started a research program for
developing UAVs,  and in 1964 is the first documented use of an UAV in a real
war scenario,  during the Vietnam War. Israel started using UAVs for 
reconnaissance and surveillance mission. As a result, Israel reported no downed 
pilot during the Lebanon War in 1982. In the present,  drones are intensively 
used in the war theaters from Afghanistan and Iraq \cite{eyes-of-the-army}.
The development of the autopilot is strongly correlated to the with the 
developed of the UAV. The company of Elmer Sperry was the first to produce
an autopilot that was able able to fly autonomous for three hours in a straight
line without being supervised by a human. By 1933 Sperry's autopilot was able to
flight on true heading and maintaining the altitude,  compared to the gyroscopic
heading of the first version.The current evolution of autopilots is in close relation with the development
of reliable communication systems. Although the first UAVs were controlled
remotely by a human operator,  they are now able to receive a flight plan
and based on that to calculate the flight path and follow it to complete the 
mission. In moder autopilots the human factor the secondary role of supervising
the system and controlling the on board equipment, like cameras and sensors. 

\section{Motivation}
\label{sec:motivation}
When I was a child I received my first toy airplane and became fascinated
by the idea of moving freely like a bird.A couple of years later I first stood 
near a MIG-21 Lancer at my fathers garrison. The passion with which the pilots 
talked about being in the air close to the clouds inspired me the love for moving 
freely in 3 dimensions.

The high number of human casualties reported in war theaters and training missions
determined me to explore the field of autonomous flying. The necessity for reducing
the loss of human lives gives autonomous great potential for evolution.

My motivation is to help create a next generation of autopilots capable of 
accomplishing difficult missions where it is the risk for a human pilot would 
not be affordable.

In Romania the UAV fields is still unexplored. The main fields where a UAV
platform would be useful are interest points detection and monitoring, border
patrol, search and rescue team and imagery intelligence.

\section{Objectives}
\label{sec:objectives}

Although a single UAV is already able do accomplishing various mission by it's
own, an interesting, and in my opinion mandatory,  field is the one of flying in
formation. A mission where the objective is to track multiple targets becomes
very hard for a single drone, thus emerging the necessity for a swarm of UAVs.
There are situation where the risk of loosing an UAV due to hostile conditions
is to high, being more affordable to deploy multiple,  cheaper UAVs in contrast
to an expensive drone. Another use case for a formation would be a search mission
where we can't equip a single aircraft with all the sensors necessary for success
and choosing to use multiple specialized drones.

Usually a human pilot is able to fly in formation using a combination of
cognitive and reactive behavior,  always making small adjustments to maintain a
coherent formation.

There are two ways that formation flight could be achieved. One would be a 
centralized method,  where all the drones report the telemetry data to a central
authority like a ground control system and the later would make the necessary 
decisions for all the involved actors and then relay the data back to the aircrafts.
Although in theory this approach could give an optimal flight path,  problems
like delay in communication and sensors reporting faulty data could jeopardize
the success of the mission. Another approach would be a decentralized method, 
inspired by swarms of animals, like ants or bees. In the second approach each 
UAV would decide what actions to execute based on the actions of the others.

The main goal of this thesis is to design an decentralized algorithm responsible
for maintaining a flight formation based on the leaders actions. The leader will
not share the flight path or mission plan with the other drones, it will share
only the current position, speed and direction. Based only on this data, the 
drones must be able to maintain a predefined flight formation. Thus each drone, 
except the leader, is modeled as an reactive agent that has the mission to 
approach the leader and mimic his actions.

The secondary goal of this paper is to design a management platform for a 
fleet of airplanes that are able to execute different missions. The platform has
the role of programming the mission for each drone,  manage the in flight
performance for each UAV and if necessary to send commands, inserted by a human
supervisor, to the drones and by this modifying the current state of the mission
execution.

The platform developed is possible thanks to a collaboration between the 
TNI\abbrev{TNI}{Teamnet International} company and ACS\abbrev{ACS}{Faculty of Automatic
Control and Computer Science}, University Politehnica Bucharest.

This thesis present a possible solution for achieving \project.

