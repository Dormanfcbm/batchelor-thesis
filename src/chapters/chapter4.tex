\chapter{Formation Flight}
\label{chapter:formation-flight}
For fling in formation a coordinate system is needed for both positioning each 
aircraft on the world and for positioning an aircraft relative to a leader. In
aviation positioning the aircraft is usually done by GPS\abbrev{GPS}{Global
Positioning System}, but this is not the best best solution for close range formations.

\section{Coordinates Systems}
\label{sec:coordinates-systems}
The Earth has a very complex and irregular shape. For mapping the position of
the aircraft a simpler mathematical model is needed. In geodesy this simpler
model is called \textit{figure of the Earth}, where the surface is usually
abstracted to the \textbf{Geoid} or \textbf{ellipsoids}. The Earth is a biaxial 
ellipsoid where the shorter axis almost matches the rotation axis \cite{gb-coordinates}.

Because the shape of the Earth doesn't fit perfectly on any ellipsoid, there
are currently multiple ellipsoids that are in use. For example the GPS uses
GRS80\abbrev{GSR80}{Geodetic Reference System 1980} \cite{gsr80} and in there
United Kingdom the Airy 1830 ellipsoid is used \cite{gb-coordinates},  being 
designed to perfectly fit Britain only. The later ellipsoid is not useful in 
other parts of the world. For measuring heights, it's necessary to define an
imaginary surface that represents the \textit{zero height}. The Geoid is used
to represent increasing heights uphill and decreasing height downhill. The
zero height shape is commonly called \textbf{sea level}.

\subsection{Latitude, Longitude,  Altitude}
\label{sub-sec:lla}
Commonly the position of the aircraft is usually done by starting with 
two angles called latitude and longitude. These two angles define a point
on the ellipsoid that fits the globe. Thus it is mandatory to know with witch
ellipsoid we are working for having any degree of certainty. On the globe
the North-South line are known as meridians,  and East-West lines are called 
parallels. To accurately position an aircraft in air a third dimension is needed.
Usually the third dimension used is altitude that is measured from \textit{sea level}.
Having a 3D Cartesian System with the z-axis oriented towards the North Pole and
the x-axis and y-axis point towards the Equator, the \textbf{Zero on longitude}
would be represented by the Z-X plane and \textbf{Zero of latitude} would be 
represented by the X-Y plane, depicted in \cite{gb-coordinates} with the following
\labelindexref{Figure}{img:lla}.
% 
\fig[scale=0.5]{src/img/lla.png}{img:lla}{Latitude, longitude representation.}
\newpage

\subsection{Earth-Centered, Earth-Fixed}
\label{sub-sec:ecef}
An alternative to the angular coordinates system is the Cartesian system. Usually
satellites use a 3D Cartesian System called ECEF\abbrev{ECEF}{Earth Center,  Earth Fixed}
Coordinates. For the Earth,  the convention is to have the origin of the system
placed at the center of the planet. The center is defined as the barycenter 
(center of mass of the earth) and respects the following mathematical formula:

\begin{equation}
\int \overrightarrow {x}\rho dx^{3} = 0
\qquad\parbox{4.0cm}{\footnotesize$\begin{aligned} 
  \overrightarrow {x} &= \text{ position vector}
  \\[-1.0ex] \rho &= \text{ density over Earth}
  \end{aligned}$}
\label{eqn:ecef-int}
\end{equation}

If \labelindexref{Equation}{eqn:ecef-int} is not satisfied, the center should
be adjusted \cite{earth-coordinates}.

As seen in \labelindexref{Figure}{img:lla}, in ECEF the z-axis points to the North
Pole,  the x-axis points to the $0^{\circ}$ meridian and leaving the y-axis to be
set so that a right handed system is created.
\subsection{Conversion}
\label{sub-sec:lla2ecef}
Conversion between an Ellipsoid System and a Cartesian System is possible 
as long as the ellipsoid parameters are known. Modern GPS systems use 
the WGS84 ellipsoid.Coincidently the ellipsoid has the same center as the ECEF
system. The ellipsoid parameters fro WGS84 are \cite{datum}:


\begin{description}
\item [Semi-major axis] \hfill \\
$a = 6378137$
\item [Semi-minor axis] \hfill \\
$b = a(1-f) = 6356752.31424518$
\item [Ellipsoid flattening] \hfill \\
$ f = \frac{1}{298.257223563}$
\item [First Eccentricity] \hfill \\
$e = \sqrt{\frac{a^{2}-b^{2}}{a^{2}}}$
\item [Second Eccentricity] \hfill \\
$e^{'} = \sqrt{\frac{a^{2}-b^{2}}{b^{2}}}$
\end{description}

\subsubsection{LLA to ECEF}
The conversion in meter from LLA to ECEF is described by the following equation
system:
\begin{equation}
\begin{cases} 
X = (N+h)\cos \varphi \cos \lambda
\\ Y = (N+h)\cos \varphi \sin \lambda
\\ Z = (\frac{b_{2}}{a_{2}}N + h) \sin \varphi
\end{cases}
\end{equation}

where

\begin {description}
\item [$\varphi$] \hfill \\ = Latitude
\item [$\lambda$] \hfill \\= Longitude
\item [$h$] \hfill \\= Height in meters above ellipsoid
\item [$N$] \hfill \\= Radius in meters of Curvature,  = $\frac{a}{\sqrt{1-e^{2} \sin^{2} \varphi}}$
\end{description}

\subsubsection{ECEF to LLA}
Converting from ECEF to LLA is more complicated but it can be achieved  using one
of the following methods \cite{datum}:

\begin{description}
\item [Iteration for $\varphi$ and $h$] \hfill \\
This method converges quickly for  $h \ll N$ starting at $h_{0}=0$
\newline
$
\begin{cases}
\lambda = \arctan \frac{X}{Y}
\\ h_{0} = 0
\\ \varphi_{0} = \arctan \frac{Z}{p(1-e^{2})}
\\ N_{i}  = \frac{a} {\sqrt{1-e^{2}\sin^{2} \varphi_{i}}}
\\ h_{i+1} = \frac{p}{\cos \varphi_{i}} - N_{i}
\end{cases}
$
\item [Closed set formula] \hfill \\
$
\begin{cases}
\lambda = \arctan \frac{X}{Y}
\\ \varphi = \arctan \frac{Z + e^{'2}b\sin^{3} \phi}{p(-e^{2}a\cos^3 \phi)}
\\ h = \frac{p}{\cos \varphi} - N
\\ p = \sqrt{X^{2} + Y^{2}}
\\ phi = \arctan \frac{Za}{pb}
\end{cases}
$
\end{description}

In this thesis the \textit{Closed set formula} was used to convert from ECEF
coordinates to GPS coordinates.

\section{Formation types}
\label{sec::formation-types}
Across the world human pilots are able to fly in formations that varying in both
shape and difficulty. The most used formations are usually formed by 3 
or 4 airplanes. 

In teams of 3 airplanes the most common formations are \textit{Aine Astern} and
\textit{V Formation} and can be seen in \labelindexref{Figure}{img:3form}.
% 
\fig[scale=0.5]{src/img/3form.png}{img:3form}{a) Line Astern Formation; b) V Formation}
\newpage

When having 4 airplanes the usual formations are: \textit{Line astern,  Box, 
Finger Right, Finger Left,  Echelon Right, Echelon Left} and can be seen in

\labelindexref{Figure}{img:4form1},  \labelindexref{Figure}{img:4form3} and
\labelindexref{Figure}{img:4form2}.
\fig[scale=0.25]{src/img/4form1.png}{img:4form1}{a)Line astern and b) Box}
\fig[scale=0.25]{src/img/4form2.png}{img:4form2}{c) Finger Right and d) Finger Left}
\fig[scale=0.25]{src/img/4form3.png}{img:4form3}{d) Finger Left and  e) Echelon Right}
\newpage

In this thesis I studied the {V Formation} and \textit{Line Astern} with different
altitudes.

\section{Entering the formation}
\label{sec:formation-entering}
\todo{Describe Formation Entering}

\section{Maintaining the formation}
\label{sec:formation-maintainig}
\todo{Describe entering the formation}
