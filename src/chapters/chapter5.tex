\chapter{Implementation details}
\label{chapter:implementation-details}

\section{FlightGear}
\label{sec:fgfs}
For the Autonomous project we have chosen an open-source, multi-platform
flight simulator called \textbf{Flight Gear Flight Simulator}. It was first
released in 1997 under \textit{GNU General Public License} by David Murr. Being
developed by a mature community in an academic environment,  FGFS\abbrev{FGFS}{Flight
Gear Flight Simulator} has reached in February 2013 version 2.10. Currently
it is being used by various universities and companies for FAA flight simulators
and research projects. Some of the universities that use FGFS are: University of
Minnesota, Department of Aerospace Engineering at The Pennsylvania State University,
University of Naples and University of Toulouse along with ATC Flight Simulator
Company Aerospace Engineering Institute from RWTH Aachen.


Being an open-source product, it offers a high degree of freedom the code being
easy to modify. Because it is written in C++ the code is cross-platform, running
on various different operating systems (eg: Windows, Linux, MAC OS). For
obtaining a higher degree of versatility, Flight Gear is able to use multiple
FDMs\abbrev{FDM}{Flight Dynamics Model}. Independent implementations like
JSBSim and YASim are built as libraries and integrated with FGFS. For these
project we have used a remote model called Rascal110 bundled with JSBSim. The
reason for using the Rascal110 model is that it has similar aerodynamics and
dimensions as the Hirrus drone, for witch we are building this autopilot.

Another reason that led to the choice of this simulator is the fact that the 
central components can be configured by a component called \textbf{Property Tree}.
Each node in the \textit{Property Tree} represent one parameter of one component
and the interaction between components can be assured by changing the values of
the parameters. The \textit{Property Tree} can be accessed in multiple ways.
One of the modes is represented by a web interfaced where the tree can be navigated
and using submit forms the values can be changed. Another is to specify the
desired parameters as command line arguments. At startup FGFS also reads an
external XML\abbrev{XML}{eXtensible  Markup Language} configuration files where
any property can be set. This allows an external user to change the current
status of the aircraft by modifying roll, speed, pitch, acceleration or even positions.

To connect to external components, FGFS uses a network communication module
that uses either TCP or UDP sockets. In combination with the \textit{Property Tree}
this modules can offer full control for outside sources. The communication module
requires an XML files that defines the file format. In the XML there are defined
an \textbf{input} and \textbf{output} formats. The \textit{output} format is 
usually used to notify the external controller of the aircraft's state, while
the \textit{input} format is used by the external controller to send commands
to control the aircraft. The messages have a C-like format and each value can
be configured to have the needed precision.

For these thesis the modules that were mostly used are the \textbf{Autopilot} and
\textbf{Route Manager}. The \textit{autopilot} has a simple interface and supports
commands to specify heading, altitude or speed.For speed and heading changes,
internally, FGFS implements PID controllers.

For ensuring a global environment for all the UAVs, we have used another component
called Flight Gear Multi-player Server. FGMS has the role of representing in the
same environment all the connected UAVs. FGMS acts as a global server where each
UAV is a client and broadcasts all the positions and flight date to all the instances.
This feature is necessary for visually observing the UAVs during the flight 
simulation.

In \labelindexref{Figure}{img:multi-uavs} can be observed from diferent perspectives 
several UAVs fling in the same environment.

\fig[scale=0.3]{src/img/multi-uavs.png}{img:multi-uavs}{Multiple UAVs in the same
environment from different perspectives}
\newpage


\section{QGroundControl}
\label{sec:qgc}

Another mandatory component for controlling a fleet an UAVs is a ground control
systems, that would allow th a graphical visualization of the mission and 
flight parameters. For this thesis the chosen ground control system is QGroundControl.

The base version that was used is v1.0.5. Originally QGC was developed
for the PIXHawk quad-rotor drone by Lorent Meier at ETH Zurich. QGC offers supports
for plotting the flight paths on a map, defining waypoints and event remote 
controlling the drones. It is developed in C++ using Qt, making it a cross-platform, 
modular architecture. QGC supports radio-copters, fixed-wing drones and even.
RC cars. For communication with external controllers QGC accepts UDP connections, 
TCP sockets, USB connections and even radio links.

Being developed for MAVs\abbrev{MAV}{Micro Air Vehicles},  QGC uses
an open-source protocol called MAVLink. MAVLink is based on a set of predefined
messages for communication between MAVs.QGC offers various widgets for facilitating
the use of MAVLink.

MAVLink uses a special message for tracking the integrity of the connected drones
called a \textbf{heartbeat}. This message is expected to come at regular intervals
of time. If this message is not received, the drone is considered incoherent
and manual actions have to be taken. Even though other messages are received from
the UAVs, without the heartbeat message, QGC sets them in a state of incoherence.

For this thesis the MAVLink protocol was bypassed and I implemented a generic
protocol based on the messages sent by FGFS. Qt's signal mechanism permitted
event generation every time a message was received from FGFS. The communication
is based on two UDP sockets (one for input and one for output) for each UAV. 
Every time a message is received from FGFS it is parsed according to the XML 
configuration used at the startup of FGFS and the position, speed, orientation
and other parameters are set and the MAV's position is rendered on the map.

An useful feature of QGC is the trail path rendered from the previous positions
of the MAV. The trail can be set to leave a marker at fixed intervals of time
or at fixed distances from one another.

QGC is used in different universities and research laboratories across the world 
like: University of Naples, French Aerospace Laboratory ONERA,  University of 
Applied Science of Hamburg.

In \labelindexref{Figure}{img:qgc}.

\fig[scale=0.3]{src/img/qgc.png}{img:qgc}{QGroundControl, monitoring the flight
path of three UAVs.}
\newpage
\section{Implementation}
\label{sec:code}
\todo{talk about code,  class-diagrams,  describe methods,  data-structures}

The formation flight module was developed using the architecture describe in 
\labelindexref{Section}{sec:architecture}.

The connection between FGFS instances and QGC, the later was modified to establish
UDP connection. For each monitored FGFS instance,  QGC opens another UDP socket
on witch it will send data back. For communication with FGFS, QGC uses a total
number of $n+1$ sockets, where n is the number of FGFS instances and the last
one is the UPD input socket. Also, I have modified the communication used
by a MAV so that it does not use MAVLink. Instead it uses a generic protocol
based on the FGFS protocol XML. Every time a message is received it is parsed
and the position, orientation, speed and other telemetry data are set in 
QGC; After this data has been set a refresh signal is emitted and the position
on the map is updated for each UAV.

Another QGC widgets was also developed in the \textit{Autonomous UAV} project
where the mission details can be configured. Using this widget a human user
can select the number of UAVs used, their flight path, the equipment for every
drone and the leader of the formation.

The formation flight module is written in C++ using the Boost library. The 
module connects to the CAN interface simulator via a TCP socket. When the module
will be introduced on the Hirrus drone, on the Raspberry PI a character device
will be crated that will read data from a CAN bus. Each drone will be equipped
with a Raspberry PI that will run a Mission Supervisor.

At startup the module reads a JSON configuration file. The configuration
file contains the number of drone and the geometry formation. Each UAV has
a relative altitude to another UAV, a position (left or right) to the possible 
leader and at what distance it should be from the other UAVs. For extensibility
reasons the JSON also contains data about the equipment.

After reading the JSON file, the data is store in a data structure called 
\textbf{uav_formation}. The \textit{uav_formation} class contains the following fields
and methods:

\begin{description}
\item [has_leader] \hfill \\ boolean flag that specifies if the formation has a UAV leader
or a virtual leader
\item [name] \hfill \\ the name of the formation; eg: V Formation
\item [drones] \hfill \\ a vector of uav_drone structures that contain data about all the
UAVs in formation.
\item [getLeader] \hfill \\ returns the uav_drone if the formation has a UAV leader
\item [getDrone] \hfill \\ returns the uav_drone with the desired name
\end{description}

\newpage
An \textbf{uav_drone} class contains the configuration for each drone of how it
should behave in formation and has the following fields:

\begin{description}
\item [name] \hfill \\ the name of the UAV drone; eg: MAV 01
\item [role] \hfill \\ specifies the role based on the UAV equipment; eg: Infrared Scanner
\item [personal_space] \hfill \\ radius for a sphere were if another UAV is detected, mild
evasive actions should be taken to avoid collision;
\item [emergency_space] \hfill \\ radius for a shpere where if another UAV is detected, 
the collision avoidance module will have maximum priority
\item [is_leader] \hfill \\ boolean marker that specifies if the current drone acts
as the leader of the formation
\item [geometry] \hfill \\ array of uav_geometry specifying the relative position of the 
current UAV to other UAVs
\end{description}

The \textbf{uav_geometry} contains data about the relative position to another drone and
it contains the following fields:

\begin{description}
\item [relative_drone] \hfill \\ the UAV to witch the geometry is relative to
\item [distance] \hfill \\ array of three values containing the height difference
between the UAVs, the heading multiply factor (used for determining how fast a UAV
will turn to enter formation), the distance between the UAVs
\end{description}

The module starts to send commands to the CAN Interface as soon as initial 
position data is available about all the UAVs. Based on the number of drones
needed in formation, the module waits until all the necessary data is received.
As long as the initial data is not present, the UAV flight is not influenced 
by this module.

Data about each UAV is kept in a data structure called \textbf{position_container}.
The \textit{position_container} acts as a hash map, keeping as key the name 
of the UAV drone and as value a pair of two \textit{uav_position} structures.
The two \textit{uav_position} instances represent the current and previous
positions for each aircraft.

The \textbf{uav_position} class is the main class of the project and is
responsible with creating commands for Flight Gear Instance. The class
contains the following fields:

\begin{description}
\item [id] \hfill \\ The id/name of the UAV
\item [latitude] \hfill \\ The GPS latitude
\item [longitude] \hfill \\ The GPS longitude
\item [altitude] \hfill \\ The altitude difference from sea level
\item [heading] \hfill \\ The true heading that the UAV drone is following
\item [speed] \hfill \\ The instantaneous speed of the UAV at the time of reporting
\item [x] \hfill \\ Value on ECEF x-axis
\item [y] \hfill \\ Value on ECEF y-axis
\item [z] \hfill \\ Value on ECEF z-axis
\end{description}
% 
This class is also responsible for converting the GPS coordinates to ECEF coordinates
and back. These methods are called: \textit{lla2ecef} and \textit{ecef2lla}. In 
this class the static method \textit{parse_uav} creates a uav_position instance
from a message received from the CAN Simulator. The received message
is parsed using a regular expression and the position, GPS position,  altitude, 
altitude heading and speed. After the ECEF data is set using the method 
\textit{lla2ecef}.
 
Based on the formation configuration and current position, a command is created 
by setting the same altitude as the leader and determining the heading 
from the current position and and the desired position. For example, if the UAV
is on the right of the leader and has to be on his left, the UAV receives a
true heading that is smaller than the leaders heading. The UAV maintains this heading
until the distance from the leader is in a desired range and is in the correct
position. On the other hand if the UAV is on the left side of the leader and
the formation specifies that it has to be on the left side, but the distance
from the leader is to big, a true heading that is greater than the leader is 
reported, until the distance is in the desired range. To determine the side on
witch the UAV resides,  from the two known position of the leader 
the flight direction is calculated and if the point is on the left or right.
In this calculations the altitude is ignored and considering that $A$ is the 
leaders previous location and $B$ is the leaders current location and $C$
is the point for witch the side is computed, the formula used is the 
cross-product is \labelindexref{Equation}{eqn:side-of-line}:

\begin{equation}
sign = \begin{vmatrix}B.x - A.x & B.y - A.y \\ C.x - A.x & C.y - A.y\end{vmatrix}
\qquad\parbox{4.0cm}{\footnotesize$\begin{aligned} 
  sign > 0  & \text{ point on left side of line}
  \\[-1.0ex] sign = 0  & \text{ point on same line}
  \\[-1.0ex] sign < 0  & \text{ point on right side of line}
  \end{aligned}$}
\label{eqn:side-of-line}
\end{equation}

In the conversion between GPS and ECEF, an computational error is introduced 
in the mathematical model,  mostly because the formulas consider the Earth Radius
of constant size. In a \textit{Line Astern} formation, this error determines
the agents to slightly move from left to right. Also this error can be seen
in calculating the distance between two UAVs. The distance computed in 
ECEF coordinates system using the Euclidean formal varies from the distance 
computed in GPS space using the Haversine formula with values ranging between
10 and 30 meters,  for distance between 100 and 200 meters.