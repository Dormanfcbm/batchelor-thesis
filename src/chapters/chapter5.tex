\chapter{Implementation details}
\label{chapter:implementation-details}

\section{FlightGear}
\label{sec:fgfs}
For the Autonomous project we have chosen an open-source, multi-platform
flight simulator called \textbf{Flight Gear Flight Simulator}. It was first
released in 1997 under \textit{GNU General Public License} by David Murr. Being
developed by a mature community in an academic environment,  FGFS\abbrev{FGFS}{Flight
Gear Flight Simulator} has reached in February 2013 version 2.10. Currently
it is being used by various universities and companies for FAA flight simulators
and research projects. Some of the universities that use FGFS are: University of
Minnesota, Department of Aerospace Engineering at The Pennsylvania State University,
University of Naples and University of Toulouse along with ATC Flight Simulator
Company Aerospace Engineering Institute from RWTH Aachen.


Being an open-source product, it offers a high degree of freedom the code being
easy to modify. Because it is written in C++ the code is cross-platform, running
on various different operating systems (eg: Windows, Linux, MAC OS). For
obtaining a higher degree of versatility, Flight Gear is able to use multiple
FDMs\abbrev{FDM}{Flight Dynamics Model}. Independent implementations like
JSBSim and YASim are built as libraries and integrated with FGFS. For these
project we have used a remote model called Rascal110 bundled with JSBSim. The
reason for using the Rascal110 model is that it has similar aerodynamics and
dimensions as the Hirrus drone, for witch we are building this autopilot.

Another reason that led to the choice of this simulator is the fact that the 
central components can be configured by a component called \textbf{Property Tree}.
Each node in the \textit{Property Tree} represent one parameter of one component
and the interaction between components can be assured by changing the values of
the parameters. The \textit{Property Tree} can be accessed in multiple ways.
One of the modes is represented by a web interfaced where the tree can be navigated
and using submit forms the values can be changed. Another is to specify the
desired parameters as command line arguments. At startup FGFS also reads an
external XML\abbrev{XML}{eXtensible  Markup Language} configuration files where
any property can be set. This allows an external user to change the current
status of the aircraft by modifying roll, speed, pitch, acceleration or even positions.

To connect to external components, FGFS uses a network communication module
that uses either TCP or UDP sockets. In combination with the \textit{Property Tree}
this modules can offer full control for outside sources. The communication module
requires an XML files that defines the file format. In the XML there are defined
an \textbf{input} and \textbf{output} formats. The \textit{output} format is 
usually used to notify the external controller of the aircraft's state, while
the \textit{input} format is used by the external controller to send commands
to control the aircraft. The messages have a C-like format and each value can
be configured to have the needed precision.

For these thesis the modules that were mostly used are the \textbf{Autopilot} and
\textbf{Route Manager}. The \textit{autopilot} has a simple interface and supports
commands to specify heading, altitude or speed.For speed and heading changes,
internally, FGFS implements PID controllers.

For ensuring a global environment for all the UAVs, we have used another component
called Flight Gear Multi-player Server. FGMS has the role of representing in the
same environment all the connected UAVs. FGMS acts as a global server where each
UAV is a client and broadcasts all the positions and flight date to all the instances.
This feature is necessary for visually observing the UAVs during the flight 
simulation.

In \labelindexref{Figure}{img:multi-uavs} can be observed from diferent perspectives 
several UAVs fling in the same environment.

\fig[scale=0.3]{src/img/multi-uavs.png}{img:multi-uavs}{Multiple UAVs in the same
environment from different perspectives}
\newpage


\section{QGroundControl}
\label{sec:fgfs}

Another mandatory component for controlling a fleet an UAVs is a ground control
systems, that would allow th a graphical visualization of the mission and 
flight parameters. For this thesis the chosen ground control system is QGroundControl.

The base version that was used is v1.0.5. Originally QGC was developed
for the PIXHawk quad-rotor drone by Lorent Meier at ETH Zurich. QGC offers supports
for plotting the flight paths on a map, defining waypoints and event remote 
controlling the drones. It is developed in C++ using Qt, making it a cross-platform, 
modular architecture. QGC supports radio-copters, fixed-wing drones and even.
RC cars. For communication with external controllers QGC accepts UDP connections, 
TCP sockets, USB connections and even radio links.

Being developed for MAVs\abbrev{MAV}{Micro Air Vehicles},  QGC uses
an open-source protocol called MAVLink. MAVLink is based on a set of predefined
messages for communication between MAVs.QGC offers various widgets for facilitating
the use of MAVLink.

MAVLink uses a special message for tracking the integrity of the connected drones
called a \textbf{heartbeat}. This message is expected to come at regular intervals
of time. If this message is not received, the drone is considered incoherent
and manual actions have to be taken. Even though other messages are received from
the UAVs, without the heartbeat message, QGC sets them in a state of incoherence.

For this thesis the MAVLink protocol was bypassed and I implemented a generic
protocol based on the messages sent by FGFS. Qt's signal mechanism permitted
event generation every time a message was received from FGFS. The communication
is based on two UDP sockets (one for input and one for output) for each UAV. 
Every time a message is received from FGFS it is parsed according to the XML 
configuration used at the startup of FGFS and the position, speed, orientation
and other parameters are set and the MAV's position is rendered on the map.

An useful feature of QGC is the trail path rendered from the previous positions
of the MAV. The trail can be set to leave a marker at fixed intervals of time
or at fixed distances from one another.

QGC is used in different universities and research laboratories across the world 
like: University of Naples, French Aerospace Laboratory ONERA,  University of 
Applied Science of Hamburg.

In \labelindexref{Figure}{img:qgc}.

\fig[scale=0.3]{src/img/qgc.png}{img:qgc}{QGroundControl, monitoring the flight
path of three UAVs.}
\newpage
\section{MY CODE}
\label{sec:fgfs}
\todo{talk about code,  class-diagrams,  describe methods,  data-structures}
